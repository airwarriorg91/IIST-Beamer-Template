\documentclass[hyperref={bookmarks=false},aspectratio=169]{beamer}
\usepackage[utf8]{inputenc}

% ---------------  Define theme and color scheme  -----------------
\usetheme[sidebarleft]{IIST}  % 3 options: minimal, sidebarleft, sidebarright

%\setbeamertemplate{footline}[frame number]

% ------------  Information on the title page  --------------------
\title[ShortTitle]
{\bfseries{Long Full Title}}

\subtitle{Subtitle}

\author[ShortAuthor1 \& ShortAuthor2]
{Author One\inst{1} \and Author Two\inst{2}}

\institute[IIST]
{
  \inst{1}
  Department\\
  Indian Insitute of Space Science and Technology
  \and
  \inst{2}
  Department\\
  Indian Insitute of Space Science and Technology
}

\date[Oct, 2024]{\today}
%------------------------------------------------------------

%------------------------------------------------------------
%The next block of commands puts the table of contents at the 
%beginning of each section and highlights the current section:

\AtBeginSection[]
{
  \begin{frame}
    \frametitle{Table of Contents}
    \tableofcontents[currentsection]
  \end{frame}
}
%------------------------------------------------------------


\begin{document}

\frame{\titlepage}  % Creates title page

%---------   table of contents after title page  ------------
\begin{frame}
\frametitle{Table of Contents}
\tableofcontents
\end{frame}
%---------------------------------------------------------


\section{Section 1}

%---------------------------------------------------------
%Changing visivility of the text
\begin{frame}
\frametitle{Slide 1}
Some itemized text...

\begin{itemize}
    \item<1-> First line comes first. 
    \item<2-> Second line joins the first in next slide.
    \item<3-> Third line joins the rest in next slide.
\end{itemize}

\end{frame}

%---------------------------------------------------------


%---------------------------------------------------------
%\begin{frame}  % Example of the \pause command
%This slide is to test mathematical formulas \pause
%
%$$E=mc^2$$ \pause
%
%as well as the ``pause'' functionality
%\end{frame}
%---------------------------------------------------------

\section{Section 2}

%---------------------------------------------------------
%Highlighting text
\begin{frame}
\frametitle{Types of Blocks}

This is a brief introduction of \alert{Blocks}.

\begin{block}{Definition}
A simple definition block.
\end{block}

\begin{alertblock}{Alert}
A simple alert block.
\end{alertblock}

\begin{examples}
A simple example block.
\end{examples}
\end{frame}
%---------------------------------------------------------


%---------------------------------------------------------
%Two columns
\begin{frame}
\frametitle{Two Columns}

\begin{columns}

\column{0.5\textwidth}
Indian Institute of Space Science and Technology (IIST), situated at Thiruvananthapuram is a Deemed to be University under Section 3 of the UGC Act 1956. IIST functions as an autonomous body under the Department of Space, Government of India. 
\column{0.5\textwidth}
The institute had the enviable reputation among all educational institutions since it had Dr A.P.J. Abdul Kalam, the renowned technocrat and former President of India as its very first Chancellor.

\small{(Reference: https://www.iist.ac.in/aboutus/institute)}

\end{columns}
\end{frame}
%---------------------------------------------------------


\end{document}